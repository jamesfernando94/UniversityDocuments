\section{Skills Gained}{
	I can comfortably say that I have gained far more skills than I expected to over the placement year. I also have found that I appear to work better in a work environment compared to an academic one. I feel that the ability to focus on one or two things at a time suites how my brain works a bit better than focusing on 4 modules at a time.
	\subsection{Programming Skills}{
		Over the past year, I have developed my programming skills a tremendous amount and not just in language specific skills but overall programming skills which can be transferred between languages.
		\subsubsection*{Programming Principals}{
			I've learnt a lot about programming principals mainly from my mentor Alex and technical lead Gihan. I have learnt a bit about design patterns like some of those listed at \url{https://sourcemaking.com/design_patterns}\cite{DesignPatterns2007}. Unsurprisingly I learnt quite a bit about creating code with unit testing in mind when I was going through the API project unit testing that. The main idea is to use interfaces where possible so that inputs can be mocked.}
		\subsubsection*{Programming Objectives}{I've mainly learnt about what the programming objectives in a commercial environment are like. These focus more on having readable maintainable code rather than focusing on the speed of the code. There is also more of a focus on making code understandable rather than adding comments to confusing code. Although I always understood the benefit of reusing code. I've learnt of the importance of being able to abstract problems to their individual parts and then what is the best way of mapping that to code.}
		\subsubsection*{Language Specific skills}{
			I've learnt a lot of C\# skills over the year such as:
			\begin{itemize}
				\item{Event Handlers}
				\item{Generics}
				\item{Tasks and Threads}
				\item{LINQ statements}
				\item{Reflection}
				\item{Lambda Functions/Delegates}
			\end{itemize}
			Although I have learnt these in C\# it shouldn't be too difficult to transfer over these skills to Java or any other language if I need to.
	}
}
	\subsection{Development Skills}{
		I've learnt quite a bit about the development process and how I can use review and QA cycles to verify code and reduce the number of bugs in it. I can also take this into my own personal development by leaving the code for a week before I review it myself. I've also learnt quite a bit about open source software development while looking at the various pieces of open source software we use. Over the year we have been using SVN for the development of the billing application and I feel that that is a very good step before learning how to use GIT. As I had tried to use GIT before but didn't really get anywhere. But SVN allows you to learn how to branch and the basics of merging and committing without the complexity of staging and pushing etc. This has meant I was able to develop this document using GIT.
	}
	\subsection{Communication skills}{I've learnt a lot about communication skills. Surprisingly I learnt a lot on my last day which I wasn't expecting as I had to go round the entire office and say goodbye to everyone. Other than that, I've learnt how to communicate and some not really important things such as email etiquette etc. I have mainly gained a lot of confidence over the year, for example, I felt that if you asked me to give a presentation on some programming aspect to a group of experienced developers I would have never imagined it would happen.}
	\subsection{Time Management Skills}{I've had to do some work on my time management although this was for a learning purpose as any deadlines were self-imposed internal deadlines which didn't really hold any weight. This was useful for me personally as it reduced stress. Although meant that my main focus for time management was on getting maximum efficiency out of mine and everyone else's time. We managed our projects using Target process\cite{targetProcess2017} this allowed us to manage the review cycles and what projects were currently being worked on. I mainly had to focus on getting my projects into a review ready state so that they could then be reviewed by one of the other developers. Target Process\cite{targetProcess2017} was useful in allowing us to split up work into tickets and then put each ticket through the review process. Which allowed us to review tickets as smaller chunks of code. I feel that this is something I can use while at University to allow me to better manage my projects and keep a track of what needs doing.
	\subsection{Planning Skills}{Although a lot of planning and proof of concept was carried out on the Turn report generator and the invoice generator projects I feel that the amount of planning carried out for my other projects were about right as it was enough planning to break down the project into manageable steps without wasting time attempting to create a structure for something without actually creating it. I feel if possible I'll bring this into my own development, however, I don't think I can use it for university work as that usually requires a lot more planning than what was done while on placement.}
}