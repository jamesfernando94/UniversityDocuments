\section{Unit Testing}{
	As one of my tasks, I was asked to implement some unit tests for the API project this was mainly to focus on the validation and population logic. 
	\subsubsection*{Moq}{
		While working on the unit tests I needed to learn how to use Moq this would enable me to test some of the existing API code. Moq is a .NET library which allows you to implement interfaces without having to define an entire class. This makes it very useful in Unit testing as it enables you to construct the input classes for a method/class without having to go through the set-up process for that input. This also taught me the importance of using Interfaces where possible when programming as it allows for easier unit testing. Also, if for example, you are writing a method it means you can abstract the function of your method away from its input meaning that it can be used in other places and with other classes.
	}
	\subsubsection*{dotCover}{
	I also used dotCover\cite{dotCover2018} to produce a coverage report before and after implementing the unit tests, although this was useful in showing my superiors the work which was carried out. It did clearly show that some parts of the code were not covered. Although you could argue that this is a valid issue to have, I feel and, I believe that my superiors agreed that this does distract you from another possible problem which is if the code which is covered is tested properly.
	} 
	\subsubsection*{Demonstration}{
		Before finishing the placement year I was asked to perform a demonstration to the development team about the best practices for Unit testing and what would be the house style going forward. I also produced a document and emailed it round to everyone so that they have some reference if they need to check what the best practices are in the future.
	}
}