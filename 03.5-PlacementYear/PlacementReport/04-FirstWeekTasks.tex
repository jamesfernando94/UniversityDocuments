\section{First Weeks Task's}{
	\subsubsection*{Read Administrator Handbook}{
		This involved reading the administrator handbook whilst playing around with a copy of the billing software to get familiar with how it worked and what all the different functions are. As although you may expect that there isn't much to the billing application I can assure you its far bigger and more complicated than most would expect.
	}
	\subsubsection*{C\# Crash Course}{
		My next task were to learn about some C\# features I heavily used \href{https://docs.microsoft.com/en-gb/dotnet/csharp/programming-guide/}{Microsoft's C\# Guide}\cite{CSProgrammingGuide2017} to learn about how to use the various features available in C\#. I used this guide to learn about generics, event handlers, LINQ statements and lambda expressions.
	}
	\subsubsection*{MVVM}{
		MVVM stands for Model View View Model. This is a design pattern for creating GUI applications. It is based on the idea of separating the display code from the logic and data. This works well in graphics packages which uses xaml for the display code for example WPF\footnote{WPF: Stands for Windows Presentation Forms is the latest display package from Microsoft to use with C\#.} and Android Development. As this allows for all of the display code to be written in a different language meaning the separation of display code and logic code is naturally enforced. It also allows for the logic to be unit tested as it should not contain any display/graphics code.
	}
	\subsubsection*{Design Patterns}{
		I also had a look at some of the design patterns listed on the website \href{https://sourcemaking.com/design_patterns}{Source Making - Design Patterns} \cite{DesignPatterns2007}
	}
}