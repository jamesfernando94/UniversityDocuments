\section{Turn Report Generator}{
	The turn report generator was one of my main projects over the year and I spend a lot of time researching and developing the project. A Turn in my sense was the inbound and outbound flight for an aircraft and all of the services carried out on that aircraft from the airports point of view. In the end, I didn't have enough time to complete it, however, I was able to start it and hopefully provide enough guidance and knowledge to Jonathan(The following replacement student) to finish the project. The turn report generator part of the billing application needed to be able to use a number of different templates depending on the which customer the turn report was for and what type of invoice it was. And then given this it needed to produce a PDF of the turn report. Originally the template was supposed to be created by a designer program produced by Damarel. However to cut the development time we decided that we should make and release the generator first and then make and release the designer. When we release the generator then we would also provide some templates created by hand so that the customer will still be able to use the generator.
	\subsubsection*{Template and Template Designer}{
		\paragraph{Template}{The Template was supposed to store information on what will be displayed in the turn report and how it would be displayed.}
		\paragraph{Designer}{The designer would then allow our customers to build the templates. In the research carried out for the designer I needed to work out how I would create this, it was requested that the designer should be of a what you see is what you get style, therefore I created a drag-drop system which would allow users to drag in the columns they wanted to be displayed on the report and drag to re-arrange them. This would make it as simple as possible for our customers to use. They would then have more options on the columns when they are dragged into the document. This drag-drop system was very complicated and a number of times I felt it was not possible to complete it. However, in the end, I did produce a proof of concept piece of software which was able to handle the desired drag-drop functionality, this did teach me that even if I think that something is not possible or more specifically it's not possible to do it cleanly and efficiently I need to look into other ways of completing the task at hand. I do feel that from a company viewpoint that all of the time and effort working on the drag-drop was a waste as a different implementation could have been carried out a lot quicker and that would not be much more difficult for the customers to understand. However, from a personal point of view, the learning experience is invaluable.}
		\paragraph{Storage}{Some parts of the templates needed to be stored in the application database, this meant that I had to design and implement some scripts to add the table/columns to an existing database and to the create scripts for the database. This meant that I had to learn a lot about the Oracle database system which is annoying because if you wish to make any small changes to a database you need to understand how all of it works and it's is quite different from MySQL which is what I had known previously. Although I did learn that in the industry they aren't as strict to the forms of normalisation as we are in university as my superiors were happy with me giving indexes to rows/items with unique keys in them. Now looking back at this design is very useful especially in the very recent case of the EU's GDPR law as this stops you from using personal customer data as a primary key as it may be deleted at the request of the customer. As we stored some information in the database this meant we had to implement a data access repository for the template. The data access repository allows us to simply call a few commands to store and retrieve the templates from the database when we give the repository a connection object.}
	}
	\subsubsection*{Turn Report Generator}{
		When it comes to generating the template we need to make sure that it is generated as a PDF then displayed and printed alongside the Invoice.
		\paragraph{PDF Generation}{
			When it came to producing the report as a PDF I was asked to do some research in the PDF packages I could use. I was given some restrictions mainly it had to be free/open source as we already had crystal reports which the customers were paying for. And, this meant we could not justify charging them for another PDF generation tool. I, therefore, did some research into free/open source PDF generation tools although this lead me down the wrong path because although a piece of software is open source that does not allow you to use it as you wish, this is because there are some terms in the licence saying that if you use this software in your own software your software needs to be open source which is a problem in a commercial environment. After this, I learnt a lot more about software licences and what I can and can't use.\todo{Add info about how the PDF generation was designed so that a different PDF generator could be switched in}
		}
		\paragraph{Integration into current application}{
			The turn report generation needed to be integrated into the billing application so that when a user clicked to view an invoice they would also be able to see the turn report as well. I conducted a small amount of research into how to integrate it and produced a couple of mock-ups on how it could look once it was all integrated.
		}
	}
}