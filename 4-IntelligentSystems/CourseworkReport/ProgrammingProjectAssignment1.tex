\documentclass[a4paper,12pt]{scrartcl}
\usepackage[utf8]{inputenc}
\usepackage[UKenglish]{isodate}
\usepackage{datetime}
\usepackage{csquotes}
\usepackage{graphicx}
\usepackage{wrapfig}
\usepackage{enumitem}
\usepackage{pdflscape}
\usepackage[toc,page]{appendix}
\usepackage{hyperref}
\usepackage{listings}
\usepackage{csvsimple}
\usepackage{booktabs}
\usepackage{longtable}
\usepackage{caption}
\usepackage{subcaption}
%\usepackage{geometry}
\usepackage[margin=2.5cm]{geometry}
\setlength{\marginparwidth}{2.5cm}
\usepackage[colorinlistoftodos]{todonotes}
\usepackage{cleveref}
\usepackage{titling}
\input{codeListingStyles.tex}

\graphicspath{ {images/} }
\usepackage[
	backend=biber,
	style=ieee,
	]{biblatex}

\addbibresource{references.bib}

\title{Programming Project Assignment 1}
\author{Candidate No: 105936}
\date{\today}

\begin{document}
	
	\begin{titlepage}
		\maketitle
	\end{titlepage}
	
	\tableofcontents
	\newpage
	\section{Introduction}
	{
		\begin{figure}[h]
			\centering
			\includegraphics[width=\textwidth]{checkersMain}
			\caption{An Image of the Starting screen for the game}
			\label{img:checkersMain}
		\end{figure}
		\Cref{img:checkersMain} shows the main screen of the game. To start with the user(you) is red and the AI is white these may be referred to dark and light in the document respectively. I probably started developing this the wrong way round developing the GUI and the game part first then adding the AI implemented with minimax later. However the alternative would to make a console game then covert it into a GUI game which probably would have produced a better output but would have taken longer to implement. I originally planed on creating a MVVM GUI application however due to time constraints and difficulty setting up the mvvm project in JavaFX with mvvmFX I decided to not follow a formal framework/structure as it would have taken more time to learn and implement. I have a number of classes in my project:
		\subsection{Main}{My main class focuses on the visuals as all of the GUI code is in this class and it also manages some of the control as well as this allows the user to interact with the board.}
		\subsection{AI}{This class contains the minimax function and the evaluation function more time could and probably should have been spend on the evaluation function. I will go on in more detail about how the evaluation function works further on in the document.}
		\subsection{Colour}{This is actually a very simple enum which is used to identify the two different players Light and Dark}
		\subsection{Draught}{This represents a draught piece and contains it's location information if it is crowned and its Colour.}
		\subsection{Game}{This manages the game and the game board state if I were redoing the project from scratch I probably would have separated out the game board data from the methods and logic as this lead to problems when implementing minimax. The game class contains methods for:
			\begin{itemize}
				\item {finding possible moves}
				\item {carrying out moves}
				\item {the successor function}
				\item {checking for a winner}
			\end{itemize}	
			This contains the majority of the logic for the game internals.
		}
		\subsection{Move}{This represents the movement of a draught and allows me to separate out some logic. And allows me to store a turn choice as an object which can be passed to other classes.}
		\subsection{Capturing Move}{This class extends the move class and allows me to store capturing moves as a different but compatible type. This is mainly because if you capture a draught and it is possible to capture another you get to continue your turn therefore it is vital to store the difference.}		
	}

	\section{Program Functionality}{
		\subsection{Game Internals}
		{
			\subsubsection{Interactive Checkers Gameplay}{
				As mentioned in the introduction I started by developing the game internals and the GUI therefore the gameplay is fairly smooth and functionally correct in terms of rules of the game. The way it works is the Main class asks for a list of possible moves from the game object and then displays them to the user for them to then decide what move to make once chosen it is run through the select move function in the game class which carries out the move for the player. For the AI a similar thing happens once the user presses the complete AI move button then the Main class gets the game and passes it to the AI which returns a move which the Main thread then passes on to the game object. 
				The user starts as the red/dark player and the AI is the white/light player.
			}
			\subsubsection{Valid State Representation}{
				The state representation used is the Game class which can be found at \cref{appedix:Game} and its made up of the following
				\paragraph{Draughts Array List}
				{
					An array list holds the draughts which are on the board currently. the draughts are stored in random order and their position information is stored in the draught class itself. The draughts also store colour information. The position information is an x and y coordinate starting from the top left of the board as that makes it easy to display in the GUI. 
				}
				\paragraph{Current Turn}
				{
					This stores the colour of the current player so the game object knows who's turn it is and therefore which moves to return when calculating them.
				}
				
				\paragraph{Selected Draught}
				{
					This is used for the completing of multiple jumps in one turn as if the player has the option to perform multiple jumps it is only possible to make moves on the draught which completed the original jump and then only if a jump is available.
				}
				\paragraph{Multi-step move}
				{
					This stores if the game is currently a part of a multi step move this means that the game can restrict the moves to only the currently selected draught.
				}
				\paragraph{Game Winner}
				{
					This stores which colour is the game winner so that Main class can display to the user which colour is the winner when the game completes. 
				}
				\paragraph{Game Complete}
				{
					This stores if the game is complete or not. and therefore display who the winner is to the user from the main class.
				}
			}
			\subsubsection{Successor function}{
				The calculation of the moves is in three steps firstly the draught calculates its possible moves given its position and if its crowned or not and removes any which aren't in bounds.
				Secondly the Game class then removes any blocked moves unless it can make a jump.
				Thirdly the \lstinline|findAllPossibleMoves()| method checks for if we are currently in a multi-step move and if so restricts the moves accordingly otherwise it gets all of the moves for the current players draughts.
				It also makes use of the \lstinline|findAllPossibleMoves()| method which works by firstly checking if we are currently on a multi-step move and if so only returning the valid capture moves from the selected draught.
				The function that I have named successor is the function in the Game class \cref{appedix:Game} which returns a following game object when passed a move.
			}
			\subsubsection{Use of Successor Function}{
				The find all moves method is used in both the game and AI classes and the find moves method is used in the main class to display the possible moves for each draught. The find all moves is used in the game class to check if there are possible moves if there aren't then the other player wins.
				It is used in the AI class as a part of the minimax method to get the children of the current game board state. The successor functions don't specifically validate the moves however the only moves either the AI or user have available to them come from the successor function therefore they must be valid.
			}
			\subsubsection{Invalid User Moves}{Invalid user moves are handled by simply not providing the function to complete incorrect moves or more accurately only providing the options to select valid moves. As there is no way for a user to select an incorrect move there is no explanation given with one exception. If the user attempts to deselect the draught during a multi-step move the program gives an error warning with an explanation saying they are only able to move the currently selected draught.
			}
			\subsubsection{Minimax Evaluation}{
				The minimax algorithm has been implemented in the AI class \cref{appendix:AI}. And uses the evaluation function \lstinline|evaluateGameBoard()| which has a very simple implementation it simply counts the number of ai draughts and subtracts the number of opponent draughts. Ideally if I had more time I would like to include if a draught was crowned or not in the calculation for example crowned draughts are worth 1.5 or 2 and going further it is also worth including non crowned draughts position up the board in the calculation although that would require a lot of fine tuning to ensure that the function does not overvalue position over pieces.
			}
			\subsubsection{Alpha Beta Pruning}
			{
				I implemented the minimax alpha beta pruning algorithm in the AI class. It is used as a part of the \lstinline|evaluateMoves()| method.
			}
			\subsubsection{Variable AI difficulty}{
				Variable AI difficultly is provided in the form of of a text box on the program where the user is able to choose a difficulty between 0-7 this represents the depth that the minimax function works to. I found that when I set the value to 8 or 9 the turns would take too long and the program would reach the 1gb JVM heap size limit therefore I have restricted the number to 7. If the user attempts to put a value higher than 7 then the program truncates it to 7.
			}
			\subsubsection{Valid AI Moves}{As mentioned earlier the AI class selects a move from moves provided from the successor function therefore are only limited to legitimate moves.}
			\subsubsection{Multi-step User Moves}{If the user has the ability to continue performing moves the application will keep the selected draught highlighted and display only the potential move. Even if there is only one move available the user still needs to select it. This has been done so that the user does not get confused when many pieces moves around. The multistep moves are checked for after every jump as a part of the \lstinline|selectMove()| method in the game class \cref{appedix:Game}. Then when the moves are requested as a part of the \lstinline|findAllPossibleMoves()| method the method restricts the moves to capturing moves of the selected draught i.e. the one which completed the last jump. Multi-step moves are also stopped if a draught gets crowned this is done by only continuing the multi-step jumps if the draught isn't on kings row or is crowned the code for this can be found as a part of the \lstinline|selectMove()| method in the game class \cref{appedix:Game}. }
			\subsubsection{Multi-step AI moves}{This is implemented in a similar way to the user multi-step. To continue the game the user needs to press the complete AI move button. This was done intentionally so that the user is able to see each of the moves made at their own pace.}
			\subsubsection{Forced Takes}{When a user has the option to make a jump they are not allowed to complete a normal move. Therefore the program only displays the jumps they are able to complete as a move if there is only one jump then they are only provided the one option this is preferred over automatically completing the move to avoid confusion on the users part. This works as a part of the \lstinline|findAllPossibleMoves()| method which gets all of the possible moves as mentioned earlier it then runs it through the \lstinline|removeNonCapturingMovesIfCapturingMoveExists(ArrayList<Move> moveList)| method. which if there is a capturing move then it restricts the moves to just capturing moves otherwise it returns all the moves.}
			\subsubsection{Automatic King Conversion}{
				Once a king reaches the king's row it gets crowned. This is done with the \lstinline|crownDraughtIfPossible()| method in the draught class \cref{appedix:Draught}. This is called on a draught after it moves. Therefore will always convert a draught correctly to display that a draught is a king a smaller circle is added to the top of the draught.
			}
		}
		\subsection{HCI/GUI}
		{
			\subsubsection{GUI Updates}{The GUI is refreshed after each turn and after the control panel is updated this is done through the \lstinline|updateDisplay()| method which redraws the display with information from the game class variable.}
			\subsubsection{Full Graphical Board Display}{
				The board displayed on the screen is made up of a gridPane each of the checker-board boxes are a coloured square these are generated in the \lstinline|generateBoard()| method in the main class \cref{appendix:Main}. The draughts are then added on top of them in the \lstinline|displayDraughtsOnBoard()| method which makes use of the \lstinline|generateDraughtVisual()| method. Then possible moves are only displayed on the correct conditions.}
			\subsubsection{Mouse Interaction}{
				The user interaction with the program is almost entirely mouse based except from setting the AI difficulty. To make a move the user needs to select a draught and then select a square to move it to. 
				This is done by making use of the \lstinline|setOnMouseClicked()| method to set an event handler for when the user clicks on a draught or a square the draught can move to. When the user clicks on a draught with possible moves the moves are displayed to the user the user can then click on any of the available moves for that draught. Once the move is selected it is passed on to the game to handle the changes, once it returns the program refreshes the display.
			}
			\subsubsection{GUI pauses on multi-step moves}{
				The GUI will only continue with the game when the user selects an action either by selecting a move or clicking the complete AI move button this was done intentionally so that the user is able to play at their own pace. This means that the program will always stop during multi-step moves and the user will have to initiate the continuation.
			}
			\subsubsection{Display of basic game rules}{
				Due to time constraints I was not able to create a page with the rules of the game however I was able to add a button to the control panel which opens a webpage with the game rules on it. This has been done in the \lstinline|generateControlPanel()| method.
			}
			\subsubsection{Possible moves display}{
				This is done by firstly showing the draughts which can be moved then, after a draught is selected, the possible squares it can move to have indicators. This is done by adding code to the \lstinline|generateDraughtVisual()| method so that the draughts have yellow borders if they have moves. then when a draught is selected possible moves are displayed using the \lstinline|displayPossibleMoves()| method.
			}
		}
		
	}
	
	
	\newpage
	\begin{appendices}
	\label{Appendix:start}
	\section{Lab Exercise 2}
	{
		\subsection{Client}
		{
			\label{appendix:ex2-Client}
			The full repo can be accessed from \url{https://os.mbed.com/users/jamesfernando/code/frdm_labex_5_1_UDPClient/}
			\lstinputlisting[language=c++]{CodeListings/Ex2/mainClient.cpp}
		}
		\subsection{Server}
		{
			\label{appendix:ex2-Server}
			The full repo can be accessed from \url{https://os.mbed.com/users/jamesfernando/code/frdm_labex_5_1_UDPServer/}
			\lstinputlisting[language=c++]{CodeListings/Ex2/mainServer.cpp}

		}
	}


\end{appendices}
	\printbibliography[heading=bibintoc,title=References]
\end{document}
