\documentclass[a4paper,12pt]{scrartcl}
\usepackage[utf8]{inputenc}
\usepackage[UKenglish]{isodate}
\usepackage{datetime}
\usepackage{csquotes}
\usepackage{graphicx}
\usepackage{wrapfig}
\usepackage{enumitem}
\usepackage{pdflscape}
\usepackage[toc,page]{appendix}
\usepackage{hyperref}
\usepackage{listings}
\usepackage{csvsimple}
\usepackage{booktabs}
\usepackage{longtable}
\usepackage{caption}
\usepackage{subcaption}
%\usepackage{geometry}
\usepackage[margin=2.5cm]{geometry}
\setlength{\marginparwidth}{2.5cm}
\usepackage[colorinlistoftodos]{todonotes}
\usepackage{cleveref}
\usepackage{titling}
\input{codeListingStyles.tex}

\graphicspath{ {images/} }
\usepackage[
	backend=biber,
	style=ieee,
	]{biblatex}

\addbibresource{references.bib}

\title{829H1 Real-Time Embedded Systems Exercise 3}
\author{Candidate No: 105936}
\date{\today}

\begin{document}
	
	\begin{titlepage}
		\maketitle
	\end{titlepage}
	
	\tableofcontents
	\newpage
	
	\section{Introduction}
	{
		This report outlines the work completed during the laboratory sessions, what equipment was used and what was learnt. Code listings for some of the created programs can be found in \cref{Appendix:start}.
	}

	\section{Equipment}
	{
		\subsection{Hardware}{
			The experiments carried out in this report were run on the Freedom-K64F prototyping board\cite{nxpproducts2014}. To use this board, we needed to connect the board to the computer using the debug port and copy across the program we wish to run and then press the reset button to load the program.
			\begin{figure}[h]
				\centering
				\includegraphics[width=0.5\textwidth]{FRDM-K64F-ANGLE}
				\caption{An Image of the Freedom-K64F Board used during experimentation\cite{nxpproducts2014}}
				\label{img:FRDM-K64F}
			\end{figure}
			We also needed to use an oscilloscope in the exercise, for this you usually need to know what pin you are outputting to, therefore I used \cref{img:pinout} to find this.
			\begin{figure}[h]
				\centering
				\includegraphics[width=0.5\textwidth]{frdm_k64f_reve4_header_pinout}
				\caption{An Image showing the pin names one the Freedom-K64F Board\cite{armlimited2015}}
				\label{img:pinout}
			\end{figure}
		}
		\subsection{Software}
		{
			To develop the experiments, I used Arm's mbed cloud IDE. This allows users to write c++ code and compile it to a device of their choice. 
		}
	}
	
	\section{Experiments}
	{
		\subsection{PWM}
		{
			For this experiment I created a PwmOut object and set the period of it to 0.01 seconds and the duty cycle to 0.5 to make identifying the period easy.
			\subsubsection{Changing the duty cycle}
			{
				I then went on to change the duty cycle to 80\% and verified my results on the oscilloscope. This shows that the wave was on for 0.008 seconds and off for 0.002 seconds at a time. The code for this can be found in \cref{appendix:ex1-1}
			}
			\subsubsection{Changing the period function}
			{
				I also changed the period function and used the \lstinline|period_ms()| and \lstinline|pulsewidth_ms()| functions and I was able to recreate the same wave with each. For the \lstinline|period_ms()| function I had to convert the time to milliseconds. For the \lstinline|pulsewidth_ms()| function I had to halve the millisecond time to get the program to output the same wave. The code for the \lstinline|pulsewidth_ms()| example can be found in \cref{appendix:ex1-2}.
			}
		}
		\subsection{Generating PWM in software}
		{
			The given example manually implements the PWM function which generates the wave on the oscilloscope.
			\subsubsection{Modifying the wait times to increase the duty cycle}
			{
				Firstly I changed the wait times so that the low duty cycle was now a medium high duty cycle of 60\% and the high duty cycle is now a very high duty cycle of 95\%. This was done by changing the wait times accordingly. I also change the time each of the duty cycles were running to 3 seconds to make it a bit quicker to test. The code for this experiment can be found in \cref{appendix:ex2-1}
			}
			\subsubsection{Modifying the program to use PwmOut}
			{
				This was quite simple as all I had to change was the \lstinline|DigitalOut| to a \lstinline|PwmOut|. Set the period in us to 1000 and then remove the for loops and set the desired duty cycle followed by a wait. The code for this can be found in \cref{appendix:ex2-2}
			}
		}
	}

	\section{Conclusion}
	{
		In conclusion, this was a useful quick introduction into how to develop using interrupts and timers also learning how to use tera term to view the console output of the board.
	}
	
	\newpage
	\begin{appendices}
	\label{Appendix:start}
	\section{Lab Exercise 2}
	{
		\subsection{Client}
		{
			\label{appendix:ex2-Client}
			The full repo can be accessed from \url{https://os.mbed.com/users/jamesfernando/code/frdm_labex_5_1_UDPClient/}
			\lstinputlisting[language=c++]{CodeListings/Ex2/mainClient.cpp}
		}
		\subsection{Server}
		{
			\label{appendix:ex2-Server}
			The full repo can be accessed from \url{https://os.mbed.com/users/jamesfernando/code/frdm_labex_5_1_UDPServer/}
			\lstinputlisting[language=c++]{CodeListings/Ex2/mainServer.cpp}

		}
	}


\end{appendices}
	\printbibliography[heading=bibintoc,title=References]
\end{document}
