\documentclass[a4paper,12pt]{scrartcl}
\usepackage[utf8]{inputenc}
\usepackage[UKenglish]{isodate}
\usepackage{datetime}
\usepackage{csquotes}
\usepackage{graphicx}
\usepackage{wrapfig}
\usepackage{enumitem}
\usepackage{pdflscape}
\usepackage[toc,page]{appendix}
\usepackage{hyperref}
\usepackage{listings}
\usepackage{csvsimple}
\usepackage{booktabs}
\usepackage{longtable}
\usepackage{caption}
\usepackage{subcaption}
%\usepackage{geometry}
\usepackage[margin=2.5cm]{geometry}
\setlength{\marginparwidth}{2.5cm}
\usepackage[colorinlistoftodos]{todonotes}
\usepackage{cleveref}
\usepackage{titling}
\input{codeListingStyles.tex}

\graphicspath{ {images/} }
\usepackage[
	backend=biber,
	style=ieee,
	]{biblatex}

\addbibresource{references.bib}

\title{829H1 Real Time Embedded Systems Exercise 1}
\author{Candidate No: 105936}
\date{\today}

\begin{document}
	
	\begin{titlepage}
		\maketitle
	\end{titlepage}
	
	\tableofcontents
	\newpage
	
	\section{Introduction}
	{
		
	}

	\section{Equipment}
	{
		\subsection{Hardware}{
			The experiments carried out in this report were run on the Freedom-K64F prototyping board\cite{nxpproducts2014}. To use this board we needed to connect the board to the computer using the debug port and copy across the program we wish to run and then press the reset button to load the program.
			\begin{figure}[h]
				\centering
				\includegraphics[width=0.5\textwidth]{FRDM-K64F-ANGLE}
				\caption{An Image of the Freedom-K64F Board used during experimentation\cite{nxpproducts2014}}
				\label{img:FRDM-K64F}
			\end{figure}
			Also in the last experiment we used and Oscilloscope to measure the voltage signals from the board to ensure that the program was working properly.
		}
		\subsection{Software}
		{
			To develop the experiments I used Arm's mbed cloud IDE. This allows users to write c++ code and compile it to a device of their choice. 
		}
	}
	
	\section{Experiments}
	{
		\subsection{Digital Output}
		{
			This section will look into programming the digital outputs of the Freedom-K64F board. More specifically I set the LED on the board to turn on and off every second.
			\subsubsection{Adding Comments}
			{
				After adding some comments to the program to describe the code better there was no observable change in the program behaviour. 
			}
			\subsubsection{Varying the Wait Parameter}
			{
				Changing the wait parameter appeared to change the time between the LED turning on and off on the board.
			}
			\subsubsection{Varying the Wait Function}
			{
				In addition to initially using the \lstinline|wait()| function I also used the \lstinline|wait_ms()| which allowed me to specify the wait time in milliseconds.
			}
			\subsubsection{Using Different LEDs}
			{
				I also attempted to use different LEDs rather than just LED2 I discovered that LED1 is red, LED2 is green and LED3 is blue. LED4 is also red for some reason although I was expecting it to be white.
			}
			\subsubsection{Using Multiple LEDs Together}
			{
				
			}
			\subsubsection{Why set the value of the LEDs to 1 and 0}
			{
				This is because the class which controls the 
			}
		}
	}

	\section{Conclusion}
	{
	
	}
	
	\newpage
	
	\printbibliography[heading=bibintoc,title=References]
\end{document}
