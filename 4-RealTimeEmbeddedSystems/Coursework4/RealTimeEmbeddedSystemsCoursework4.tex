\documentclass[a4paper,12pt]{scrartcl}
\usepackage[utf8]{inputenc}
\usepackage[UKenglish]{isodate}
\usepackage{datetime}
\usepackage{csquotes}
\usepackage{graphicx}
\usepackage{wrapfig}
\usepackage{enumitem}
\usepackage{pdflscape}
\usepackage[toc,page]{appendix}
\usepackage{hyperref}
\usepackage{listings}
\usepackage{csvsimple}
\usepackage{booktabs}
\usepackage{longtable}
\usepackage{caption}
\usepackage{subcaption}
%\usepackage{geometry}
\usepackage[margin=2.5cm]{geometry}
\setlength{\marginparwidth}{2.5cm}
\usepackage[colorinlistoftodos]{todonotes}
\usepackage{cleveref}
\usepackage{titling}
\input{codeListingStyles.tex}

\graphicspath{ {images/} }
\usepackage[
	backend=biber,
	style=ieee,
	]{biblatex}

\addbibresource{references.bib}

\title{829H1 Real-Time Embedded Systems Exercise 4}
\author{Candidate No: 105936}
\date{\today}

\begin{document}
	
	\begin{titlepage}
		\maketitle
	\end{titlepage}
	
	\tableofcontents
	\newpage
	
	\section{Introduction}
	{
		This report focuses on using serial communications on the board and shows the work completed during the laboratory sessions and what was learnt. Code listings for some of the created programs can be found in \cref{Appendix:start}.
	}
	
	\section{Experiments}
	{
		\subsection{Serial Communications}
		{
			For this experiment I created a SPI master object which allowed me to output data using serial methods on three pins, these pins were PTD2 which was the MOSI, PTD3 the MISO and PTD1 which was the clock. 
			\begin{figure}[h]
				\centering
				\includegraphics[width=0.5\textwidth]{Ex1/Task1-Edited}
				\caption{An Image showing the output on the oscilloscope when running code from \cref{appendix:ex1-1}}
				\label{img:Ex1-Task1}
			\end{figure}
			\cref{img:Ex1-Task1} demonstrates the output of sending a word out using serial communication. Here the blue line represents the clock and red the signals.
			
			\subsubsection{Using Different SPI Modes}
			{
				This changes the clock signal as shown in the 4 images below.
				\begin{figure}[h]
					\centering
					\includegraphics[width=0.45\textwidth]{Ex1/mode0}
					\includegraphics[width=0.45\textwidth]{Ex1/mode1}
					\includegraphics[width=0.45\textwidth]{Ex1/mode2}
					\includegraphics[width=0.45\textwidth]{Ex1/mode3}
					\caption{Images showing the same word being written using different clock modes}
					\label{img:DifferentClockModes}
				\end{figure}
			}
		}
		\subsection{Linking Two Boards together}
		{

		}
	}

	\section{Conclusion}
	{
		In conclusion, this was a useful quick introduction into how to develop using interrupts and timers also learning how to use tera term to view the console output of the board.
	}
	
	\newpage
	\begin{appendices}
	\label{Appendix:start}
	\section{Lab Exercise 2}
	{
		\subsection{Client}
		{
			\label{appendix:ex2-Client}
			The full repo can be accessed from \url{https://os.mbed.com/users/jamesfernando/code/frdm_labex_5_1_UDPClient/}
			\lstinputlisting[language=c++]{CodeListings/Ex2/mainClient.cpp}
		}
		\subsection{Server}
		{
			\label{appendix:ex2-Server}
			The full repo can be accessed from \url{https://os.mbed.com/users/jamesfernando/code/frdm_labex_5_1_UDPServer/}
			\lstinputlisting[language=c++]{CodeListings/Ex2/mainServer.cpp}

		}
	}


\end{appendices}
	\printbibliography[heading=bibintoc,title=References]
\end{document}
