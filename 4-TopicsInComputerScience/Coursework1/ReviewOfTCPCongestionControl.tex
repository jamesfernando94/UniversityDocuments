\documentclass[a4paper,12pt]{scrartcl}
\usepackage[utf8]{inputenc}
\usepackage[UKenglish]{isodate}
\usepackage{datetime}
\usepackage{csquotes}
\usepackage{graphicx}
\usepackage{wrapfig}
\usepackage{enumitem}
\usepackage{pdflscape}
\usepackage[toc,page]{appendix}
\usepackage{hyperref}
\usepackage{listings}
\usepackage{csvsimple}
\usepackage{booktabs}
\usepackage{longtable}
\usepackage{caption}
\usepackage{subcaption}
%\usepackage{geometry}
\usepackage[margin=2.5cm]{geometry}
\setlength{\marginparwidth}{2.5cm}
\usepackage[colorinlistoftodos]{todonotes}
\usepackage{cleveref}
\usepackage{titling}
\input{codeListingStyles.tex}

\graphicspath{ {images/} }
\usepackage[
	backend=biber,
	style=ieee,
	]{biblatex}

\addbibresource{references.bib}

\title{Review of TCP Congestion Control Papers}
\author{Candidate No: 105936}
\date{\today}

\begin{document}
	
	\begin{titlepage}
		\maketitle
	\end{titlepage}
	
	\tableofcontents
	\newpage
	
	\section{Congestion Avoidance and Control\cite{JacobsonCongestAvoidanceControlArticle}}
	{
		This paper was the first to suggest implementing congestion control in TCP. Van Jacobson discovered that the internet was having congestion collapses. This is where throughput would drop to a fraction of what it should be this is because the network would keep dropping packets as there isn't enough bandwidth and the computers would keep resending the failed packets. Jacobson then goes on to outline the congestion control solution which would introduce many features which are still in use such as congestion windows and slow-start.
	}
	\section{Simulation-based Comparisons of Tahoe,Reno, and SACK TCP\cite{FallFloydTahoeRenoSack}}
	{
		
	}
	\section{CUBIC: a new TCP-friendly high-speed TCP variant\cite{HaRheeXuCubic}}
	{
		This paper proposed a novel new way of implementing BIC(Binary Increase Control) TCP using a cubic function. This provided two benefits the width of the curve could be adjusted in the cubic implementation which means that for connections with small round trip times it wouldn't follow the curve too quickly. Also implementing the binary increase algorithm could be quite computationally expensive where as implementing the cubic function would be very easy.
		\begin{figure}[h]
			\centering
			\includegraphics[width=0.75\textwidth]{BICVsCubicRTT}
			\caption{An Image showing the difference between CUBIC and BIC with a respect to round trip times\cite{deawookim2015}}
			\label{fig:BICVsCubicRTT}
		\end{figure}
	}
	\section{Improving Datacenter Performance and Robustness with Multipath TCP\cite{RaiciuBarrePluntkeGreenhalghWischikHandleyMultipathTCPArticle}}
	{
		
	}
	\newpage
	
	\printbibliography[heading=bibintoc,title=References]
\end{document}
