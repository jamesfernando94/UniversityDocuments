\documentclass[a4paper,12pt]{scrartcl}
\usepackage[utf8]{inputenc}
\usepackage[UKenglish]{isodate}
\usepackage{datetime}
\usepackage{csquotes}
\usepackage{graphicx}
\usepackage{wrapfig}
\usepackage{enumitem}
\usepackage{pdflscape}
\usepackage[toc,page]{appendix}
\usepackage{hyperref}
\usepackage{listings}
\usepackage{csvsimple}
\usepackage{booktabs}
\usepackage{longtable}
\usepackage{caption}
\usepackage{subcaption}
%\usepackage{geometry}
\usepackage[margin=2.5cm]{geometry}
\setlength{\marginparwidth}{2.5cm}
\usepackage[colorinlistoftodos]{todonotes}
\usepackage{cleveref}
\usepackage{titling}
\input{codeListingStyles.tex}

\graphicspath{ {images/} }
\usepackage[
	backend=biber,
	style=ieee,
	]{biblatex}

\addbibresource{references.bib}

\title{Software verification for real world applications}
\author{Candidate No: 105936}
\date{\today}

\begin{document}
	
	\begin{titlepage}
		\maketitle
	\end{titlepage}
	
	\tableofcontents
	\newpage
	\section{Introduction}
	{
		Generally software verification is a very interesting topic in research at the moment, however it is currently limited to the field of researchers and it it is only really used as a part of demonstration software and only as a part of verifiable languages. Therefore this paper will look into how the software verification techniques can be applied to applications designed for use in the real world. This will obviously have advantages as it guarantees code free of fatal runtime errors and reduces the likelihood of other errors.
	}

	\section{Current Problem}
	{
		Currently if we compare software verification to unit testing as a form of guarantee against bugs we see that unit testing is far more popular and there are far more frameworks available to use for unit testing than there are for verifying software. Also, if you look at the current Computer Science Degree at the University of Sussex unit testing is taught in the first year and software verification is not taught until the masters level. This paper will look into the different software verification techniques and why they are not used more often.
	}
	
	\section{Existing Software verification techniques}
	{
		There are three types of static analysis are Abstract static analysis, model checking and, bounded model checking.\cite{DSilva2008} 
		\subsection{Abstract Static Analysis}
		{
			This verification technique was introduced in \cite{Cousot1977} as a way of reducing runtime errors, they saw that strong typing was a start in reducing run time errors and then went on to look into how to make pointers safer. 
		}
		\subsection{Model Checking}
		
		\subsection{Bounded Model Checking}
	}
	
	\section{Existing Verifiable Languages}
	{
		
	}

	\section{How to bring software verification into the mainstream}
	{
		\todo{Come up with an idea about how to bring software verification into the mainstream}
		\cite{Flanagan2002} shows how to add static checking to Java programs but may not go into how to allow for some parts to be checked and others not to be checked. Possibly through the use of multiple languages for example the .NET enviroment has many languages which can be compiled into .NET libraries and used interchangeably and there are also a number of languages which can be compiled to run on the JVM. Therefore is there an example of a language which compiles into .NET or JVM for use with other libraries.
	}
	
	\section{Conclusion}
	{
	
	}
	
	\newpage
	
	\printbibliography[heading=bibintoc,title=References]
\end{document}
