\documentclass[a4paper,12pt]{scrartcl}
\usepackage[utf8]{inputenc}
\usepackage[UKenglish]{isodate}
\usepackage{csquotes}
\usepackage{graphicx}
\usepackage{wrapfig}
\usepackage{enumitem}
\usepackage{pdflscape}
\usepackage[toc,page]{appendix}
\usepackage{geometry}
\usepackage{hyperref}
\usepackage{cleveref}
\usepackage{listings}
\usepackage{csvsimple}
\usepackage{booktabs}
\usepackage{longtable}
\usepackage{caption}
\usepackage{subcaption}
\usepackage[colorinlistoftodos]{todonotes}
\usepackage[british]{babel}
\usepackage{float}
%\usepackage[margin=1in]{geometry}
\usepackage{listings}
\usepackage{color}
 
\definecolor{codegreen}{rgb}{0,0.6,0}
\definecolor{codegray}{rgb}{0.5,0.5,0.5}
\definecolor{codepurple}{rgb}{0.58,0,0.82}
\definecolor{backcolour}{rgb}{0.95,0.95,0.92}
 
\lstdefinestyle{mystyle}{
	language=PHP,
    backgroundcolor=\color{backcolour},   
    commentstyle=\color{codegray},
    keywordstyle=\color{magenta},
    numberstyle=\tiny\color{codegray},
    stringstyle=\color{codegreen},
    basicstyle=\footnotesize,
    breakatwhitespace=false,         
    breaklines=true,                 
    captionpos=b,                    
    keepspaces=true,                 
    numbers=left,                    
    numbersep=5pt,                  
    showspaces=false,                
    showstringspaces=false,
    showtabs=false,                  
    tabsize=3,
    morekeywords={ new, __halt_compiler, abstract, and, array, as, break, callable, case, catch, class, clone, const, continue, declare, default, die, do, echo, else, elseif, empty, enddeclare, endfor, endforeach, endif, endswitch, endwhile, eval, exit, extends, final, for, foreach, function, global, goto, if, implements, include, include_once, instanceof, insteadof, interface, isset, list, namespace, new, or, print, private, protected, public, require, require_once, return, static, switch, throw, trait, try, unset, use, var, while, xor}
}

\lstset{language=Java,
  showspaces=false,
  showtabs=false,
  breaklines=true,
  showstringspaces=false,
  breakatwhitespace=true,
  commentstyle=\color{pgreen},
  keywordstyle=\color{pblue},
  stringstyle=\color{pred},
  basicstyle=\ttfamily,
  moredelim=[il][\textcolor{pgrey}]{$$},
  moredelim=[is][\textcolor{pgrey}]{\%\%}{\%\%}
}
 
\lstset{style=mystyle}

\graphicspath{ {images/} }
\usepackage[
	backend=biber,
	style=ieee,
	]{biblatex}

\addbibresource{references.bib}

\title{Survey of Software verification for real world applications}
\author{Candidate No: 105936}
\date{\today}

\begin{document}
	
	\begin{titlepage}
		\maketitle
	\end{titlepage}
	
	\tableofcontents
	\newpage
	\section{Introduction}
	{
		Generally software verification is a very interesting topic in research at the moment, however it is currently limited to the field of researchers and it it is only really used as a part of demonstration software and only as a part of verifiable languages. Therefore this paper will look into how the software verification techniques can be applied to applications designed for use in the real world. This will obviously have advantages as it guarantees code free of fatal runtime errors and reduces the likelihood of other errors.
	}

	\section{Current Problem}
	{
		Currently if we compare software verification to unit testing as a form of guarantee against bugs we see that unit testing is far more popular and there are far more frameworks available to use for unit testing than there are for verifying software. Also, if you look at the current Computer Science Degree at the University of Sussex unit testing is taught in the first year and software verification is not taught until the masters level. This paper will look into the different software verification techniques and why they are not used more often.
	}
	
	\section{Existing Software verification techniques}
	{
		There are three types of static analysis are Abstract static analysis, model checking and, bounded model checking.\cite{DSilva2008} 
		\subsection{Abstract Static Analysis}
		{
			This verification technique was introduced in \cite{Cousot1977} as a way of reducing runtime errors, they saw that strong typing was a start in reducing run time errors and then went on to look into how to make pointers safer. Static analysis allows for the analysis of a program without actually executing the program. The way in which it works is by computing a superset of possible values for each stage of the program. You can then look at the sets for example of the set for a loop parameter is infinite then you can tell that it may not exit\todo{Check that this is correct} or you can check the set of values for a divisor and if one is zero then you may have a divide by zero error. Obviously if these are not in the set of values you can guarantee that the program does not contain divide by zero errors. which is obviously a very useful trait.
		}
		\subsection{Model Checking}
		{
		
		}
		\subsection{Bounded Model Checking}
	}
	
	\section{Existing Verification Tools}
	{
		As a part of looking into how to include software verification in real world applications we need to look at how they can be used in mainstream languages therefore in the following sections I have described a number of tools and what features they support in verifying programs written in mainstream languages.
	}

	\section{How to bring software verification into the mainstream}
	{
		\todo{Come up with an idea about how to bring software verification into the mainstream}
		\cite{Flanagan2002} shows how to add static checking to Java programs but may not go into how to allow for some parts to be checked and others not to be checked. Possibly through the use of multiple languages for example the .NET enviroment has many languages which can be compiled into .NET libraries and used interchangeably and there are also a number of languages which can be compiled to run on the JVM. Therefore is there an example of a language which compiles into .NET or JVM for use with other libraries.
	}
	
	\section{Conclusion}
	{
	
	}
	
	\newpage
	
	\printbibliography[heading=bibintoc,title=References]
\end{document}
