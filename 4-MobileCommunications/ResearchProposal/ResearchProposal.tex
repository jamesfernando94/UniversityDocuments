\documentclass[a4paper,12pt]{scrartcl}
\usepackage[utf8]{inputenc}
\usepackage[UKenglish]{isodate}
\usepackage{datetime}
\usepackage{csquotes}
\usepackage{graphicx}
\usepackage{wrapfig}
\usepackage{enumitem}
\usepackage{pdflscape}
\usepackage[toc,page]{appendix}
\usepackage{hyperref}
\usepackage{listings}
\usepackage{csvsimple}
\usepackage{booktabs}
\usepackage{longtable}
\usepackage{caption}
\usepackage{subcaption}
%\usepackage{geometry}
\usepackage[margin=2.5cm]{geometry}
\setlength{\marginparwidth}{2.5cm}
\usepackage[colorinlistoftodos]{todonotes}
\usepackage{cleveref}
\usepackage{titling}
\input{codeListingStyles.tex}

\graphicspath{ {images/} }
\usepackage[
	backend=biber,
	style=ieee,
	]{biblatex}

\addbibresource{references.bib}

\title{Research Proposal on the LTE Radio Access Network(E-UTRAN)}
\author{James Fernando}
\date{\today}

\begin{document}
	
	\begin{titlepage}
		\maketitle
	\end{titlepage}
	
	\tableofcontents
	\newpage	
{
	\section{Introduction}{
		
	}
	\section{LTE - Radio Access Network (E-UTRAN) works}
	{
		The E-UTRAN is made up of multiple eNodeB bases stations and manages the connection between the mobiles and the evolved packet core. The eNB uses the air interface to communicate with the mobile devices. It also manages the handover of the mobile devices to other eNBs.
		\url{https://www.tutorialspoint.com/lte/lte_network_architecture.htm}
		\url{https://rc.library.uta.edu/uta-ir/bitstream/handle/10106/5546/Shah_uta_2502M_10889.pdf?sequence=1}
		\url{ftp://ftp.3gpp.org/inbox/2008_web_files/LTA_Paper.pdf}
	}
	\section{Inter Cell Radio Resource Management}
	{
		https://jwcn-eurasipjournals.springeropen.com/articles/10.1155/2010/531347
	}
	\section{RB Control}
	{
		
	}
	\section{Connection Movability Control}
	{
		
	}
	\section{Radio Admission Control}
	{
		
	}
	\section{eNB measurement Configuration \& Provision}
	{
		
	}
	\section{Dynamic Resource Allocation(Scheduler)}
	{
	
	}
	\section{Conclusion}
	{
		
	}
	
	
	\newpage
	
	\printbibliography[heading=bibintoc,title=References]
\end{document}
