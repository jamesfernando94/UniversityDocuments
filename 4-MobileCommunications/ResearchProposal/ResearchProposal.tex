\documentclass[a4paper,12pt]{scrartcl}
\usepackage[utf8]{inputenc}
\usepackage[UKenglish]{isodate}
\usepackage{csquotes}
\usepackage{graphicx}
\usepackage{wrapfig}
\usepackage{enumitem}
\usepackage{pdflscape}
\usepackage[toc,page]{appendix}
\usepackage{geometry}
\usepackage{hyperref}
\usepackage{cleveref}
\usepackage{listings}
\usepackage{csvsimple}
\usepackage{booktabs}
\usepackage{longtable}
\usepackage{caption}
\usepackage{subcaption}
\usepackage[colorinlistoftodos]{todonotes}
\usepackage[british]{babel}
\usepackage{float}
%\usepackage[margin=1in]{geometry}
\usepackage{listings}
\usepackage{color}
 
\definecolor{codegreen}{rgb}{0,0.6,0}
\definecolor{codegray}{rgb}{0.5,0.5,0.5}
\definecolor{codepurple}{rgb}{0.58,0,0.82}
\definecolor{backcolour}{rgb}{0.95,0.95,0.92}
 
\lstdefinestyle{mystyle}{
	language=PHP,
    backgroundcolor=\color{backcolour},   
    commentstyle=\color{codegray},
    keywordstyle=\color{magenta},
    numberstyle=\tiny\color{codegray},
    stringstyle=\color{codegreen},
    basicstyle=\footnotesize,
    breakatwhitespace=false,         
    breaklines=true,                 
    captionpos=b,                    
    keepspaces=true,                 
    numbers=left,                    
    numbersep=5pt,                  
    showspaces=false,                
    showstringspaces=false,
    showtabs=false,                  
    tabsize=3,
    morekeywords={ new, __halt_compiler, abstract, and, array, as, break, callable, case, catch, class, clone, const, continue, declare, default, die, do, echo, else, elseif, empty, enddeclare, endfor, endforeach, endif, endswitch, endwhile, eval, exit, extends, final, for, foreach, function, global, goto, if, implements, include, include_once, instanceof, insteadof, interface, isset, list, namespace, new, or, print, private, protected, public, require, require_once, return, static, switch, throw, trait, try, unset, use, var, while, xor}
}

\lstset{language=Java,
  showspaces=false,
  showtabs=false,
  breaklines=true,
  showstringspaces=false,
  breakatwhitespace=true,
  commentstyle=\color{pgreen},
  keywordstyle=\color{pblue},
  stringstyle=\color{pred},
  basicstyle=\ttfamily,
  moredelim=[il][\textcolor{pgrey}]{$$},
  moredelim=[is][\textcolor{pgrey}]{\%\%}{\%\%}
}
 
\lstset{style=mystyle}

\graphicspath{ {images/} }
\usepackage[
	backend=biber,
	style=ieee,
	]{biblatex}

\addbibresource{references.bib}

\title{Research Proposal on the LTE Radio Access Network(E-UTRAN)}
\author{James Fernando}
\date{\today}

\begin{document}
	
	\begin{titlepage}
		\maketitle
	\end{titlepage}
	
	\tableofcontents
	\newpage	
	\section{What to do}
	{
		\subsection{Plan}{
			I expect to provide a fairly abstract description of the E-UTRAN system. After which I will go though briefly explaining the sections on Research, Implementation and Deployment of the E-UTRAN system. before then picking a more specific part of the E-UTRAN system and critiquing it.
		
		\subsection{Provide a Brief explanation of how the current LTE - Radio Access Network (E-UTRAN) works}
		{
			This will provide a non-technical overview of the implementation of LTE and attempt to explain to a layman how it works.
%			The E-UTRAN is made up of multiple eNodeB bases stations and manages the connection between the mobiles and the evolved packet core. The eNB uses the air interface to communicate with the mobile devices. It also manages the handover of the mobile devices to other eNBs.
			\url{https://www.tutorialspoint.com/lte/lte_network_architecture.htm}
			\url{https://rc.library.uta.edu/uta-ir/bitstream/handle/10106/5546/Shah_uta_2502M_10889.pdf?sequence=1}
		}
		\subsection{Research on LTE}
		{
			I will provide some details on the research which went into creating LTE and possibly some other interesting details about the early stages of LTE
			\url{ftp://ftp.3gpp.org/inbox/2008_web_files/LTA_Paper.pdf}
		}
		\subsection{Implementation of LTE}
		{
			I will provide some more details on the implementation and go into a more specific section of the part I intend on critiquing. 
			\url{http://www.artizanetworks.com/resources/tutorials/eut_arc.html}
		}
		\subsection{Deployment of LTE}
		{
			Provide some information about LTE adoption and what changes needed to be made to implemented systems to make it 4G compatible  
			\url{https://www.researchgate.net/publication/228998229_Handover_Scenario_and_Procedure_in_LTE-based_Femtocell_Networks}
		}
		\subsection{Critical Review of LTE}
		{
			Look at something which could be improved in LTE and check if there is any research showing why they use the current implementation. One possibility could be the connection handling because currently there are two states one which handles the mobile devices which are in the cell but are not using the channels for connection and one for devices using the connection. this seems like something which has not been updated for current phone usage. Because previously most phones would not be using a connection until the user decided to make a call or send a text however now most phones are always using a data connection therefore this original state may not be needed.  
		}
	}
	
	\newpage
	
	\printbibliography[heading=bibintoc,title=References]
\end{document}
