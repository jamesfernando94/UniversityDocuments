\documentclass[a4paper,12pt]{scrartcl}
\usepackage[utf8]{inputenc}
\usepackage[UKenglish]{isodate}
\usepackage{datetime}
\usepackage{csquotes}
\usepackage{graphicx}
\usepackage{wrapfig}
\usepackage{enumitem}
\usepackage{pdflscape}
\usepackage[toc,page]{appendix}
\usepackage{hyperref}
\usepackage{listings}
\usepackage{csvsimple}
\usepackage{booktabs}
\usepackage{longtable}
\usepackage{caption}
\usepackage{subcaption}
%\usepackage{geometry}
\usepackage[margin=2.5cm]{geometry}
\setlength{\marginparwidth}{2.5cm}
\usepackage[colorinlistoftodos]{todonotes}
\usepackage{cleveref}
\usepackage{titling}
\input{codeListingStyles.tex}

\graphicspath{ {images/} }
\usepackage[
	backend=biber,
	style=ieee,
	]{biblatex}

\addbibresource{references.bib}

\title{Research Proposal on the LTE Radio Access Network(E-UTRAN)}
\author{James Fernando}
\date{\today}

\begin{document}
	
	\begin{titlepage}
		\maketitle
	\end{titlepage}
	
	\tableofcontents
	\newpage	
	\section{What to do}
	{
		\subsection{Provide a Brief explanation of how the current LTE - Radio Access Network (E-UTRAN) works}
		{
			The E-UTRAN is made up of multiple eNodeB bases stations and manages the connection between the mobiles and the evolved packet core. 
			\url{https://www.tutorialspoint.com/lte/lte_network_architecture.htm}
			\url{https://rc.library.uta.edu/uta-ir/bitstream/handle/10106/5546/Shah_uta_2502M_10889.pdf?sequence=1}
		}
		\subsection{Research on LTE}
		{
			\url{ftp://ftp.3gpp.org/inbox/2008_web_files/LTA_Paper.pdf}
		}
		\subsection{Implementation of LTE}
		{
			\url{http://www.artizanetworks.com/resources/tutorials/eut_arc.html}
		}
		\subsection{Deployment of LTE}
		{
			\url{https://www.researchgate.net/publication/228998229_Handover_Scenario_and_Procedure_in_LTE-based_Femtocell_Networks}
		}
		\subsection{Critical Review of LTE}
		{
		
		}
	}
	
	\newpage
	
	\printbibliography[heading=bibintoc,title=References]
\end{document}
